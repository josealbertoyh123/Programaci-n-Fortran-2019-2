\documentclass{article}
\usepackage[utf8]{inputenc}

\title{\textbf{Movimiento de Proyectiles}}
\author{José Alberto Romero Lagarda}
\date{30 de agosto de 2019}

\begin{document}
\maketitle
\large
\textbf{Movimiento parabólico} (también conocido como 
\textbf{tiro parabólico}) es el nombre que comúnmente 
se le da a aquel movimiento de un cuerpo que traza una 
trayectoria parabólica. Los conceptos de "tiro" y 
"proyectil" aparacen a razón 
de la aparición del movimiento en lanzamientos "libres",
es decir, hechos bajo un campo gravitatorio uniforme y sin 
ninguna fuerza que ofrezca resistencia.

Cabe mencionar que la trayectoria trazada por el proyectil
no es exactamente la de una parábola, sino de parcialmente 
una elipse. Esto es: cuando un cuerpo es lanzado no directamente
al centro de gravedad del cuerpo al que está sometido, el objeto 
dibuja una trayectoria elíptica mientras se acerca a él. En este
caso, el suelo impide que la trayectoria sea trazada completamente
y lo que resulta es sólo un trozo de elipse que, para su estudio, 
es tan parecido al de una parábola que resulta más sencillo 
considerarlo como tal.

Una característica básica y esencial para el estudio 
de estemovimiento , es el hecho de que puede ser analizado como
un movimiento en dos dimensiones: por un lado un movimiento
rectilíneo uniforme (MRU), y por otro, un tiro vertical.

Otras características que hay tomar en cuenta son las siguientes:
\begin{itemize}
    \item Conociendo la velocidad de salida (inicial), el ángulo de inclinación inicial 
    y la diferencia de alturas (entre salida y llegada) se conocerá toda la trayectoria.
    \item Los ángulos de salida y llegada son iguales (siempre que 
    la altura de salida y de llegada sean iguales).
    \item La mayor distancia cubierta o alcance se logra con ángulos de salida de 45º.
    \item Para lograr la mayor distancia fijado el ángulo 
el factor más importante es la velocidad.
\item Un cuerpo que se deja caer libremente y otro que es lanzado 
horizontalmente desde la misma altura tardan lo mismo en llegar al suelo.
\item La independencia de la masa en la caída libre y el lanzamiento 
vertical es igual de válida en los movimientos parabólicos.
\item El tiempo que tarda en alcanzar su altura máxima es el mismo 
tiempo que tarda en recorrer la mitad de su distancia horizontal, 
es decir, el tiempo total necesario para alcanzar la distancia horizontal 
máxima es el doble del tiempo empleado en alcanzar su altura máxima.
\end{itemize}

Las ecuaciones para estye movimiento son:

1.- $\textbf{v}=v_{0}cos\phi\textbf{i}+v_{0}sin\phi\textbf{j}$

2.- $\textbf{a}=-g\textbf{j}$
\vspace{0.2cm}

donde:

$v$ es el módulo de la velocidad final.

$v_{0}$ es el módulo de la velocidad inicial.

$\phi$ es el ángulo de la velocidad inicial sobre la horizontal.

$g$ es la aceleración de la gravedad.

$\textbf{i},\textbf{j}$ son dos vectores (vectores unitarios) en el plano.

\vspace{0.5cm}

\LARGE\textbf{Referencias}
\vspace{0.2cm}

\large1.-https://es.wikipedia.org/wiki/Movimiento\_parabólico
\vspace{0.5cm}


A continuación se presente la tabla respectiva a la 
parte primera de la actividad 2.
\vspace{0.2cm}

\begin{tabular}{|l|l|l|l|}
\hline
\alpha ($^\circ$) & V_{0} (m/s) & Tiempo (s) & x_{max} (m) \\ \hline
20         & 30         & 2.09400105 & 59.0315208 \\ \hline
25         & 30         & 2.58745861 & 70.3510132 \\ \hline
30         & 30         & 3.06122446 & 79.5329437 \\ \hline
35         & 30         & 3.51169276 & 86.2983017 \\ \hline
40         & 30         & 3.93543434 & 90.4415283 \\ \hline
45         & 30         & 4.32922506 & 91.8367310 \\ \hline
50         & 30         & 4.69006777 & 90.4415283 \\ \hline
55         & 30         & 5.01521683 & 86.2983017 \\ \hline
60         & 30         & 5.30219650 & 79.5329437 \\ \hline
65         & 30         & 5.54882288 & 70.3510132 \\ \hline
\end{tabular}



\end{document}

